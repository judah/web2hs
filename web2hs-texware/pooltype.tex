\input webmac
% This program by D. E. Knuth is not copyrighted and can be used freely.
% Version 1 was implemented in June 1982.
% Slight changes were made in October, 1982, for version 0.6 of TeX.
% Version 2 (July 1983) is consistent with TeX version 0.999.
% Version 3 (September 1989) is consistent with 8-bit TeX.

% Here is TeX material that gets inserted after \input webmac
\def\hang{\hangindent 3em\indent\ignorespaces}
\font\ninerm=cmr9
\let\mc=\ninerm % medium caps for names like SAIL
\def\PASCAL{Pascal}

\def\(#1){} % this is used to make section names sort themselves better
\def\9#1{} % this is used for sort keys in the index

\def\title{POOL\lowercase{type}}
\def\contentspagenumber{101}
\def\topofcontents{\null
  \def\titlepage{F} % include headline on the contents page
  \def\rheader{\mainfont\hfil \contentspagenumber}
  \vfill
  \centerline{\titlefont The {\ttitlefont POOLtype} processor}
  \vskip 15pt
  \centerline{(Version 3, September 1989)}
  \vfill}
\def\botofcontents{\vfill
  \centerline{\hsize 5in\baselineskip9pt
    \vbox{\ninerm\noindent
    The preparation of this report
    was supported in part by the National Science
    Foundation under grants IST-8201926 and MCS-8300984,
    and by the System Development Foundation. `\TeX' is a
    trademark of the American Mathematical Society.}}}
\pageno=\contentspagenumber \advance\pageno by 1


\N1.  Introduction.
The \.{POOLtype} utility program converts string pool files output
by \.{TANGLE} into a slightly more symbolic format that may be useful
when \.{TANGLE}d programs are being debugged.

It's a pretty trivial routine, but people may want to try transporting
this program before they get up enough courage to tackle \TeX\ itself.
The first 256 strings are treated as \TeX\ treats them, using routines
copied from \TeX82.

\fi

\M2. \.{POOLtype} is written entirely in standard \PASCAL, except that it has
to do some slightly system-dependent character code conversion on input
and output. The input is read from \\{pool\_file}, and the output is written
on \\{output}. If the input is erroneous, the \\{output} file will describe
the error.

\Y\P\4\&{program}\1\  \37$\\{POOLtype}(\\{pool\_file},\39\\{output})$;\6
\4\&{label} \379999;\C{this labels the end of the program}\6
\4\&{type} \37\X5:Types in the outer block\X\6
\4\&{var} \37\X7:Globals in the outer block\X\6
\4\&{procedure}\1\  \37\\{initialize};\C{this procedure gets things started
properly}\6
\4\&{var} \37\X6:Local variables for initialization\X\2\6
\&{begin} \37\X8:Set initial values of key variables\X\6
\&{end};\par
\fi

\M3. Here are some macros for common programming idioms.

\Y\P\D \37$\\{incr}(\#)\S\#\K\#+1$\C{increase a variable by unity}\par
\P\D \37$\\{decr}(\#)\S\#\K\#-1$\C{decrease a variable by unity}\par
\P\D \37$\\{do\_nothing}\S$\C{empty statement}\par
\fi

\N4.  The character set.
(The following material is copied verbatim from \TeX82.
Thus, the same system-dependent changes should be made to both programs.)

In order to make \TeX\ readily portable between a wide variety of
computers, all of its input text is converted to an internal eight-bit
code that includes standard ASCII, the ``American Standard Code for
Information Interchange.''  This conversion is done immediately when each
character is read in. Conversely, characters are converted from ASCII to
the user's external representation just before they are output to a
text file.

Such an internal code is relevant to users of \TeX\ primarily because it
governs the positions of characters in the fonts. For example, the
character `\.A' has ASCII code $65=\O{101}$, and when \TeX\ typesets
this letter it specifies character number 65 in the current font.
If that font actually has `\.A' in a different position, \TeX\ doesn't
know what the real position is; the program that does the actual printing from
\TeX's device-independent files is responsible for converting from ASCII to
a particular font encoding.

\TeX's internal code is relevant also with respect to constants
that begin with a reverse apostrophe; and it provides an index to the
\.{\\catcode}, \.{\\mathcode}, \.{\\uccode}, \.{\\lccode}, and \.{\\delcode}
tables.

\fi

\M5. Characters of text that have been converted to \TeX's internal form
are said to be of type \\{ASCII\_code}, which is a subrange of the integers.

\Y\P$\4\X5:Types in the outer block\X\S$\6
$\\{ASCII\_code}=0\to255$;\C{eight-bit numbers}\par
\U2.\fi

\M6. The original \PASCAL\ compiler was designed in the late 60s, when six-bit
character sets were common, so it did not make provision for lowercase
letters. Nowadays, of course, we need to deal with both capital and small
letters in a convenient way, especially in a program for typesetting;
so the present specification of \TeX\ has been written under the assumption
that the \PASCAL\ compiler and run-time system permit the use of text files
with more than 64 distinguishable characters. More precisely, we assume that
the character set contains at least the letters and symbols associated
with ASCII codes \O{40} through \O{176}; all of these characters are now
available on most computer terminals.

Since we are dealing with more characters than were present in the first
\PASCAL\ compilers, we have to decide what to call the associated data
type. Some \PASCAL s use the original name \\{char} for the
characters in text files, even though there now are more than 64 such
characters, while other \PASCAL s consider \\{char} to be a 64-element
subrange of a larger data type that has some other name.

In order to accommodate this difference, we shall use the name \\{text\_char}
to stand for the data type of the characters that are converted to and
from \\{ASCII\_code} when they are input and output. We shall also assume
that \\{text\_char} consists of the elements $\\{chr}(\\{first\_text\_char})$
through
$\\{chr}(\\{last\_text\_char})$, inclusive. The following definitions should be
adjusted if necessary.

\Y\P\D \37$\\{text\_char}\S\\{char}$\C{the data type of characters in text
files}\par
\P\D \37$\\{first\_text\_char}=0$\C{ordinal number of the smallest element of %
\\{text\_char}}\par
\P\D \37$\\{last\_text\_char}=255$\C{ordinal number of the largest element of %
\\{text\_char}}\par
\Y\P$\4\X6:Local variables for initialization\X\S$\6
\4\|i: \37\\{integer};\par
\U2.\fi

\M7. The \TeX\ processor converts between ASCII code and
the user's external character set by means of arrays \\{xord} and \\{xchr}
that are analogous to \PASCAL's \\{ord} and \\{chr} functions.

\Y\P$\4\X7:Globals in the outer block\X\S$\6
\4\\{xord}: \37\&{array} $[\\{text\_char}]$ \1\&{of}\5
\\{ASCII\_code};\C{specifies conversion of input characters}\2\6
\4\\{xchr}: \37\&{array} $[\\{ASCII\_code}]$ \1\&{of}\5
\\{text\_char};\C{specifies conversion of output characters}\2\par
\As12, 13\ETs18.
\U2.\fi

\M8. Since we are assuming that our \PASCAL\ system is able to read and
write the visible characters of standard ASCII (although not
necessarily using the ASCII codes to represent them), the following
assignment statements initialize the standard part of the \\{xchr} array
properly, without needing any system-dependent changes. On the other
hand, it is possible to implement \TeX\ with less complete character
sets, and in such cases it will be necessary to change something here.

\Y\P$\4\X8:Set initial values of key variables\X\S$\6
$\\{xchr}[\O{40}]\K\.{\'\ \'}$;\5
$\\{xchr}[\O{41}]\K\.{\'!\'}$;\5
$\\{xchr}[\O{42}]\K\.{\'"\'}$;\5
$\\{xchr}[\O{43}]\K\.{\'\#\'}$;\5
$\\{xchr}[\O{44}]\K\.{\'\$\'}$;\5
$\\{xchr}[\O{45}]\K\.{\'\%\'}$;\5
$\\{xchr}[\O{46}]\K\.{\'\&\'}$;\5
$\\{xchr}[\O{47}]\K\.{\'\'}\.{\'\'}$;\6
$\\{xchr}[\O{50}]\K\.{\'(\'}$;\5
$\\{xchr}[\O{51}]\K\.{\')\'}$;\5
$\\{xchr}[\O{52}]\K\.{\'*\'}$;\5
$\\{xchr}[\O{53}]\K\.{\'+\'}$;\5
$\\{xchr}[\O{54}]\K\.{\',\'}$;\5
$\\{xchr}[\O{55}]\K\.{\'-\'}$;\5
$\\{xchr}[\O{56}]\K\.{\'.\'}$;\5
$\\{xchr}[\O{57}]\K\.{\'/\'}$;\6
$\\{xchr}[\O{60}]\K\.{\'0\'}$;\5
$\\{xchr}[\O{61}]\K\.{\'1\'}$;\5
$\\{xchr}[\O{62}]\K\.{\'2\'}$;\5
$\\{xchr}[\O{63}]\K\.{\'3\'}$;\5
$\\{xchr}[\O{64}]\K\.{\'4\'}$;\5
$\\{xchr}[\O{65}]\K\.{\'5\'}$;\5
$\\{xchr}[\O{66}]\K\.{\'6\'}$;\5
$\\{xchr}[\O{67}]\K\.{\'7\'}$;\6
$\\{xchr}[\O{70}]\K\.{\'8\'}$;\5
$\\{xchr}[\O{71}]\K\.{\'9\'}$;\5
$\\{xchr}[\O{72}]\K\.{\':\'}$;\5
$\\{xchr}[\O{73}]\K\.{\';\'}$;\5
$\\{xchr}[\O{74}]\K\.{\'<\'}$;\5
$\\{xchr}[\O{75}]\K\.{\'=\'}$;\5
$\\{xchr}[\O{76}]\K\.{\'>\'}$;\5
$\\{xchr}[\O{77}]\K\.{\'?\'}$;\6
$\\{xchr}[\O{100}]\K\.{\'@\'}$;\5
$\\{xchr}[\O{101}]\K\.{\'A\'}$;\5
$\\{xchr}[\O{102}]\K\.{\'B\'}$;\5
$\\{xchr}[\O{103}]\K\.{\'C\'}$;\5
$\\{xchr}[\O{104}]\K\.{\'D\'}$;\5
$\\{xchr}[\O{105}]\K\.{\'E\'}$;\5
$\\{xchr}[\O{106}]\K\.{\'F\'}$;\5
$\\{xchr}[\O{107}]\K\.{\'G\'}$;\6
$\\{xchr}[\O{110}]\K\.{\'H\'}$;\5
$\\{xchr}[\O{111}]\K\.{\'I\'}$;\5
$\\{xchr}[\O{112}]\K\.{\'J\'}$;\5
$\\{xchr}[\O{113}]\K\.{\'K\'}$;\5
$\\{xchr}[\O{114}]\K\.{\'L\'}$;\5
$\\{xchr}[\O{115}]\K\.{\'M\'}$;\5
$\\{xchr}[\O{116}]\K\.{\'N\'}$;\5
$\\{xchr}[\O{117}]\K\.{\'O\'}$;\6
$\\{xchr}[\O{120}]\K\.{\'P\'}$;\5
$\\{xchr}[\O{121}]\K\.{\'Q\'}$;\5
$\\{xchr}[\O{122}]\K\.{\'R\'}$;\5
$\\{xchr}[\O{123}]\K\.{\'S\'}$;\5
$\\{xchr}[\O{124}]\K\.{\'T\'}$;\5
$\\{xchr}[\O{125}]\K\.{\'U\'}$;\5
$\\{xchr}[\O{126}]\K\.{\'V\'}$;\5
$\\{xchr}[\O{127}]\K\.{\'W\'}$;\6
$\\{xchr}[\O{130}]\K\.{\'X\'}$;\5
$\\{xchr}[\O{131}]\K\.{\'Y\'}$;\5
$\\{xchr}[\O{132}]\K\.{\'Z\'}$;\5
$\\{xchr}[\O{133}]\K\.{\'[\'}$;\5
$\\{xchr}[\O{134}]\K\.{\'\\\'}$;\5
$\\{xchr}[\O{135}]\K\.{\']\'}$;\5
$\\{xchr}[\O{136}]\K\.{\'\^\'}$;\5
$\\{xchr}[\O{137}]\K\.{\'\_\'}$;\6
$\\{xchr}[\O{140}]\K\.{\'\`\'}$;\5
$\\{xchr}[\O{141}]\K\.{\'a\'}$;\5
$\\{xchr}[\O{142}]\K\.{\'b\'}$;\5
$\\{xchr}[\O{143}]\K\.{\'c\'}$;\5
$\\{xchr}[\O{144}]\K\.{\'d\'}$;\5
$\\{xchr}[\O{145}]\K\.{\'e\'}$;\5
$\\{xchr}[\O{146}]\K\.{\'f\'}$;\5
$\\{xchr}[\O{147}]\K\.{\'g\'}$;\6
$\\{xchr}[\O{150}]\K\.{\'h\'}$;\5
$\\{xchr}[\O{151}]\K\.{\'i\'}$;\5
$\\{xchr}[\O{152}]\K\.{\'j\'}$;\5
$\\{xchr}[\O{153}]\K\.{\'k\'}$;\5
$\\{xchr}[\O{154}]\K\.{\'l\'}$;\5
$\\{xchr}[\O{155}]\K\.{\'m\'}$;\5
$\\{xchr}[\O{156}]\K\.{\'n\'}$;\5
$\\{xchr}[\O{157}]\K\.{\'o\'}$;\6
$\\{xchr}[\O{160}]\K\.{\'p\'}$;\5
$\\{xchr}[\O{161}]\K\.{\'q\'}$;\5
$\\{xchr}[\O{162}]\K\.{\'r\'}$;\5
$\\{xchr}[\O{163}]\K\.{\'s\'}$;\5
$\\{xchr}[\O{164}]\K\.{\'t\'}$;\5
$\\{xchr}[\O{165}]\K\.{\'u\'}$;\5
$\\{xchr}[\O{166}]\K\.{\'v\'}$;\5
$\\{xchr}[\O{167}]\K\.{\'w\'}$;\6
$\\{xchr}[\O{170}]\K\.{\'x\'}$;\5
$\\{xchr}[\O{171}]\K\.{\'y\'}$;\5
$\\{xchr}[\O{172}]\K\.{\'z\'}$;\5
$\\{xchr}[\O{173}]\K\.{\'\{\'}$;\5
$\\{xchr}[\O{174}]\K\.{\'|\'}$;\5
$\\{xchr}[\O{175}]\K\.{\'\}\'}$;\5
$\\{xchr}[\O{176}]\K\.{\'\~\'}$;\par
\As10, 11\ETs14.
\U2.\fi

\M9. Some of the ASCII codes without visible characters have been given
symbolic
names in this program because they are used with a special meaning.

\Y\P\D \37$\\{null\_code}=\O{0}$\C{ASCII code that might disappear}\par
\P\D \37$\\{carriage\_return}=\O{15}$\C{ASCII code used at end of line}\par
\P\D \37$\\{invalid\_code}=\O{177}$\C{ASCII code that many systems prohibit in
text files}\par
\fi

\M10. The ASCII code is ``standard'' only to a certain extent, since many
computer installations have found it advantageous to have ready access
to more than 94 printing characters. Appendix~C of {\sl The \TeX book\/}
gives a complete specification of the intended correspondence between
characters and \TeX's internal representation.

If \TeX\ is being used
on a garden-variety \PASCAL\ for which only standard ASCII
codes will appear in the input and output files, it doesn't really matter
what codes are specified in $\\{xchr}[0\to\O{37}]$, but the safest policy is to
blank everything out by using the code shown below.

However, other settings of \\{xchr} will make \TeX\ more friendly on
computers that have an extended character set, so that users can type things
like `\.^^Z' instead of `\.{\\ne}'. People with extended character sets can
assign codes arbitrarily, giving an \\{xchr} equivalent to whatever
characters the users of \TeX\ are allowed to have in their input files.
It is best to make the codes correspond to the intended interpretations as
shown in Appendix~C whenever possible; but this is not necessary. For
example, in countries with an alphabet of more than 26 letters, it is
usually best to map the additional letters into codes less than~\O{40}.
To get the most ``permissive'' character set, change \.{\'\ \'} on the
right of these assignment statements to $\\{chr}(\|i)$.

\Y\P$\4\X8:Set initial values of key variables\X\mathrel{+}\S$\6
\&{for} $\|i\K0\mathrel{\&{to}}\O{37}$ \1\&{do}\5
$\\{xchr}[\|i]\K\.{\'\ \'}$;\2\6
\&{for} $\|i\K\O{177}\mathrel{\&{to}}\O{377}$ \1\&{do}\5
$\\{xchr}[\|i]\K\.{\'\ \'}$;\2\par
\fi

\M11. The following system-independent code makes the \\{xord} array contain a
suitable inverse to the information in \\{xchr}. Note that if $\\{xchr}[\|i]=%
\\{xchr}[\|j]$
where $\|i<\|j<\O{177}$, the value of $\\{xord}[\\{xchr}[\|i]]$ will turn out
to be
\|j or more; hence, standard ASCII code numbers will be used instead of
codes below \O{40} in case there is a coincidence.

\Y\P$\4\X8:Set initial values of key variables\X\mathrel{+}\S$\6
\&{for} $\|i\K\\{first\_text\_char}\mathrel{\&{to}}\\{last\_text\_char}$ \1%
\&{do}\5
$\\{xord}[\\{chr}(\|i)]\K\\{invalid\_code}$;\2\6
\&{for} $\|i\K\O{200}\mathrel{\&{to}}\O{377}$ \1\&{do}\5
$\\{xord}[\\{xchr}[\|i]]\K\|i$;\2\6
\&{for} $\|i\K0\mathrel{\&{to}}\O{176}$ \1\&{do}\5
$\\{xord}[\\{xchr}[\|i]]\K\|i$;\2\par
\fi

\N12.  String handling.
(The following material is copied from the \\{get\_strings\_started} procedure
of \TeX82, with slight changes.)

\Y\P$\4\X7:Globals in the outer block\X\mathrel{+}\S$\6
\4$\|k,\39\|l$: \37$0\to255$;\C{small indices or counters}\6
\4$\|m,\39\|n$: \37\\{text\_char};\C{characters input from \\{pool\_file}}\6
\4\|s: \37\\{integer};\C{number of strings treated so far}\par
\fi

\M13. The global variable \\{count} keeps track of the total number of
characters
in strings.

\Y\P$\4\X7:Globals in the outer block\X\mathrel{+}\S$\6
\4\\{count}: \37\\{integer};\C{how long the string pool is, so far}\par
\fi

\M14. \P$\X8:Set initial values of key variables\X\mathrel{+}\S$\6
$\\{count}\K0$;\par
\fi

\M15. This is the main program, where \.{POOLtype} starts and ends.

\Y\P\D \37$\\{abort}(\#)\S$\1\6
\&{begin} \37$\\{write\_ln}(\#)$;\5
\&{goto} \379999;\6
\&{end}\2\par
\Y\P\&{begin} \37\\{initialize};\6
\X16:Make the first 256 strings\X;\6
$\|s\K256$;\6
\X19:Read the other strings from the \.{POOL} file, or give an error message
and abort\X;\6
$\\{write\_ln}(\.{\'(\'},\39\\{count}:1,\39\.{\'\ characters\ in\ all.)\'})$;\6
\49999: \37\&{end}.\par
\fi

\M16. \P\D \37$\\{lc\_hex}(\#)\S\|l\K\#$;\6
\&{if} $\|l<10$ \1\&{then}\5
$\|l\K\|l+\.{"0"}$\ \&{else} $\|l\K\|l-10+\.{"a"}$\2\par
\Y\P$\4\X16:Make the first 256 strings\X\S$\6
\&{for} $\|k\K0\mathrel{\&{to}}255$ \1\&{do}\6
\&{begin} \37$\\{write}(\|k:3,\39\.{\':\ "\'})$;\5
$\|l\K\|k$;\6
\&{if} $(\X17:Character \|k cannot be printed\X)$ \1\&{then}\6
\&{begin} \37$\\{write}(\\{xchr}[\.{"\^"}],\39\\{xchr}[\.{"\^"}])$;\6
\&{if} $\|k<\O{100}$ \1\&{then}\5
$\|l\K\|k+\O{100}$\6
\4\&{else} \&{if} $\|k<\O{200}$ \1\&{then}\5
$\|l\K\|k-\O{100}$\6
\4\&{else} \&{begin} \37$\\{lc\_hex}(\|k\mathbin{\&{div}}16)$;\5
$\\{write}(\\{xchr}[\|l])$;\5
$\\{lc\_hex}(\|k\mathbin{\&{mod}}16)$;\5
$\\{incr}(\\{count})$;\6
\&{end};\2\2\6
$\\{count}\K\\{count}+2$;\6
\&{end};\2\6
\&{if} $\|l=\.{""}\.{""}$ \1\&{then}\5
$\\{write}(\\{xchr}[\|l],\39\\{xchr}[\|l])$\6
\4\&{else} $\\{write}(\\{xchr}[\|l])$;\2\6
$\\{incr}(\\{count})$;\5
$\\{write\_ln}(\.{\'"\'})$;\6
\&{end}\2\par
\U15.\fi

\M17. The first 128 strings will contain 95 standard ASCII characters, and the
other 33 characters will be printed in three-symbol form like `\.{\^\^A}'
unless a system-dependent change is made here. Installations that have
an extended character set, where for example $\\{xchr}[\O{32}]=\hbox{\.{\'^^Z%
\'}}$,
would like string \O{32} to be the single character \O{32} instead of the
three characters \O{136}, \O{136}, \O{132} (\.{\^\^Z}). On the other hand,
even people with an extended character set will want to represent string
\O{15} by \.{\^\^M}, since \O{15} is \\{carriage\_return}; the idea is to
produce visible strings instead of tabs or line-feeds or carriage-returns
or bell-rings or characters that are treated anomalously in text files.

Unprintable characters of codes 128--255 are, similarly, rendered
\.{\^\^80}--\.{\^\^ff}.

The boolean expression defined here should be \\{true} unless \TeX\
internal code number~\|k corresponds to a non-troublesome visible
symbol in the local character set.  An appropriate formula for the
extended character set recommended in {\sl The \TeX book\/} would, for
example, be `$\|k\in[0,\O{10}\to\O{12},\O{14},\O{15},\O{33},\O{177}\to%
\O{377}]$'.
If character \|k cannot be printed, and $\|k<\O{200}$, then character $\|k+%
\O{100}$ or
$\|k-\O{100}$ must be printable; moreover, ASCII codes $[\O{41}\to\O{46},\O{60}%
\to\O{71},\O{141}\to\O{146},\O{160}\to\O{171}]$ must be printable.
Thus, at least 80 printable characters are needed.

\Y\P$\4\X17:Character \|k cannot be printed\X\S$\6
$(\|k<\.{"\ "})\V(\|k>\.{"\~"})$\par
\U16.\fi

\M18. When the \.{WEB} system program called \.{TANGLE} processes a source
file,
it outputs a \PASCAL\ program and also a string pool file. The present
program reads the latter file, where each string appears as a two-digit decimal
length followed by the string itself, and the information is output with its
associated index number. The strings are surrounded by double-quote marks;
double-quotes in the string itself are repeated.

\Y\P$\4\X7:Globals in the outer block\X\mathrel{+}\S$\6
\4\\{pool\_file}: \37\&{packed} \37\&{file} \1\&{of}\5
\\{text\_char};\C{the string-pool file output by \.{TANGLE}}\2\6
\4\\{xsum}: \37\\{boolean};\C{has the check sum been found?}\par
\fi

\M19. \P$\X19:Read the other strings from the \.{POOL} file, or give an error
message and abort\X\S$\6
$\\{reset}(\\{pool\_file})$;\5
$\\{xsum}\K\\{false}$;\6
\&{if} $\\{eof}(\\{pool\_file})$ \1\&{then}\5
$\\{abort}(\.{\'!\ I\ can\'}\.{\'t\ read\ the\ POOL\ file.\'})$;\2\6
\1\&{repeat} \37\X20:Read one string, but abort if there are problems\X;\6
\4\&{until}\5
\\{xsum};\2\6
\&{if} $\R\\{eof}(\\{pool\_file})$ \1\&{then}\5
$\\{abort}(\.{\'!\ There\'}\.{\'s\ junk\ after\ the\ check\ sum\'})$\2\par
\U15.\fi

\M20. \P$\X20:Read one string, but abort if there are problems\X\S$\6
\&{if} $\\{eof}(\\{pool\_file})$ \1\&{then}\5
$\\{abort}(\.{\'!\ POOL\ file\ contained\ no\ check\ sum\'})$;\2\6
$\\{read}(\\{pool\_file},\39\|m,\39\|n)$;\C{read two digits of string length}\6
\&{if} $\|m\I\.{\'*\'}$ \1\&{then}\6
\&{begin} \37\&{if} $(\\{xord}[\|m]<\.{"0"})\V(\\{xord}[\|m]>\.{"9"})\V(%
\\{xord}[\|n]<\.{"0"})\V(\\{xord}[\|n]>\.{"9"})$ \1\&{then}\5
$\\{abort}(\.{\'!\ POOL\ line\ doesn\'}\.{\'t\ begin\ with\ two\ digits\'})$;\2%
\6
$\|l\K\\{xord}[\|m]\ast10+\\{xord}[\|n]-\.{"0"}\ast11$;\C{compute the length}\6
$\\{write}(\|s:3,\39\.{\':\ "\'})$;\5
$\\{count}\K\\{count}+\|l$;\6
\&{for} $\|k\K1\mathrel{\&{to}}\|l$ \1\&{do}\6
\&{begin} \37\&{if} $\\{eoln}(\\{pool\_file})$ \1\&{then}\6
\&{begin} \37$\\{write\_ln}(\.{\'"\'})$;\5
$\\{abort}(\.{\'!\ That\ POOL\ line\ was\ too\ short\'})$;\6
\&{end};\2\6
$\\{read}(\\{pool\_file},\39\|m)$;\5
$\\{write}(\\{xchr}[\\{xord}[\|m]])$;\6
\&{if} $\\{xord}[\|m]=\.{""}\.{""}$ \1\&{then}\5
$\\{write}(\\{xchr}[\.{""}\.{""}])$;\2\6
\&{end};\2\6
$\\{write\_ln}(\.{\'"\'})$;\5
$\\{incr}(\|s)$;\6
\&{end}\6
\4\&{else} $\\{xsum}\K\\{true}$;\2\6
$\\{read\_ln}(\\{pool\_file})$\par
\U19.\fi

\N21.  System-dependent changes.
This section should be replaced, if necessary, by changes to the program
that are necessary to make \.{POOLtype} work at a particular installation.
It is usually best to design your change file so that all changes to
previous sections preserve the section numbering; then everybody's version
will be consistent with the printed program. More extensive changes,
which introduce new sections, can be inserted here; then only the index
itself will get a new section number.

\fi

\N22.  Index.
Indications of system dependencies appear here together with the section
numbers
where each ident\-i\-fier is used.
\fi


\inx
\:\\{abort}, \[15], 19, 20.
\:{ASCII code}, 4.
\:\\{ASCII\_code}, \[5], 6, 7.
\:\\{boolean}, 18.
\:\\{carriage\_return}, \[9], 17.
\:\\{char}, 6.
\:{character set dependencies}, 10, 17.
\:\\{chr}, 6, 7, 10, 11.
\:\\{count}, \[13], 14, 15, 16, 20.
\:\\{decr}, \[3].
\:\\{do\_nothing}, \[3].
\:\\{eof}, 19, 20.
\:\\{eoln}, 20.
\:\\{false}, 19.
\:\\{first\_text\_char}, \[6], 11.
\:\|{i}, \[6].
\:\\{incr}, \[3], 16, 20.
\:\\{initialize}, \[2], 15.
\:\\{integer}, 6, 12, 13.
\:\\{invalid\_code}, \[9], 11.
\:\|{k}, \[12].
\:\|{l}, \[12].
\:\\{last\_text\_char}, \[6], 11.
\:\\{lc\_hex}, \[16].
\:\|{m}, \[12].
\:\|{n}, \[12].
\:\\{null\_code}, \[9].
\:\\{ord}, 7.
\:\\{output}, \[2].
\:\\{pool\_file}, \[2], 12, \[18], 19, 20.
\:\\{POOLtype}, \[2].
\:\\{read}, 20.
\:\\{read\_ln}, 20.
\:\\{reset}, 19.
\:\|{s}, \[12].
\:{system dependencies}, 2, 6, 8, 10, 17, 21.
\:\9{TeXbook}{\sl The \TeX book}, 10, 17.
\:\\{text\_char}, \[6], 7, 12, 18.
\:\\{true}, 17, 20.
\:\\{write}, 16, 20.
\:\\{write\_ln}, 15, 16, 20.
\:\\{xchr}, \[7], 8, 10, 11, 16, 17, 20.
\:\\{xord}, \[7], 11, 20.
\:\\{xsum}, \[18], 19, 20.
\fin
\:\X17:Character \|k cannot be printed\X
\U16.
\:\X7, 12, 13, 18:Globals in the outer block\X
\U2.
\:\X6:Local variables for initialization\X
\U2.
\:\X16:Make the first 256 strings\X
\U15.
\:\X20:Read one string, but abort if there are problems\X
\U19.
\:\X19:Read the other strings from the \.{POOL} file, or give an error message
and abort\X
\U15.
\:\X8, 10, 11, 14:Set initial values of key variables\X
\U2.
\:\X5:Types in the outer block\X
\U2.
\con
